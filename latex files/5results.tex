\chapter{Results}
\label{chapter:results}

The aim of the study was to examine the effect of three design conditions on outcomes. The design conditions were three different interaction schemes: the robot signs and says the word in question (abbreviated in tables as ``Sign only"); the robot signs, says the word in question and shows an image of the word (abbreviated in tables as ``Image"); and the robot signs, says the word in question and flashes lights (abbreviated in tables as ``Light").

First, quantitative analysis of the child's responses to the robot's signs and the child's gaze direction are presented. Second, the qualitative analysis of the surveys conducted with the children and their companions is presented.

As a reminder, the hypotheses tested in the experiments were:
\vspace{3mm}

\noindent\textbf{H1: Children will imitate signs performed by the robot}
\vspace{3mm}

\noindent\textbf{H2: The design condition will affect the success of imitation}
\vspace{3mm}

\noindent\textbf{H3: The design condition will affect the child's attention focus}
\vspace{3mm}


%%%%%%%%%%
%%%%%%%%%%


\section{Quantitative analysis}

The Wilcoxon signed-rank test is used to examine the success rate of the children's imitations of the signs. The Wilcoxon test is a non-parametric statistical hypothesis test, and is used to perform a one-sample test to determine whether the median number of imitations is statistically significantly different from zero \cite{wilcoxon1945individual}. The Friedman test was used to analyze the potential effect of the three different design conditions on both the child's ability to sign accurately and the child's gaze direction. The Friedman test is a non-parametric test, and is used to detect differences in outcomes across multiple test attempts \cite{friedman1937use}. An $\alpha$ level of $0.05$ was used for all statistical tests.


\subsection{Successful imitation of signs}

\begin{table}
  \centering
  \renewcommand{\arraystretch}{1.2}
  \begin{tabular}{|p{3cm}|c|c|c|c|}
    \hline
    \textbf{Child} &
    \textbf{Sign only} &
    \textbf{Image} &
    \textbf{Light} & 
    \textbf{All}\\
    \hline
    A & 2 & 2 & 1 & \textbf{5}\\ \hline
    B & 2 & 3 & 1 & \textbf{6}\\ \hline
    C & 2 & 3 & 2 & \textbf{7} \\ \hline
    D & 0 & 0 & 0 & \textbf{0} \\ \hline
    E & 3 & 3 & 3 & \textbf{9} \\ \hline
    F & 0 & 0 & 0 & \textbf{0} \\ \hline
    G & 0 & 0 & 0 & \textbf{0} \\ \hline
    H & 0 & 1 & 3 & \textbf{4} \\ \hline
    I & 3 & 3 & 3 & \textbf{9} \\ \hline
    J & 2 & 2 & 1 & \textbf{5} \\ \hline

  \end{tabular}
 
  \caption{Successful imitations of each child in different design conditions, and combined successes for all design conditions.}
  \label{table:imitation}
\end{table}

Table \ref{table:imitation} depicts the success of the children imitating the robot across different design conditions. Considering the ``All" column in table \ref{table:imitation}, 7 out of 10 children performed a non-zero number of imitations of the robot's signs. A one-sample Wilcoxon signed rank test reveals that the median number of total successful repetitions was significantly greater than zero ($p = 0.01101$). Thus H1 can be accepted – children were able to imitate the robot's signs. However, a Friedman test showed no significant effect of design condition on successful imitations, $\chi^2 = 0.46154, p = 0.7939$.

No significant effect of design condition was discovered on the outcomes of incorrect signing ($\chi^2 = 0.66667, p = 0.7165$), no reaction to the signing prompt ($\chi^2 = 2, p = 0.3679$), or correct with human assistance ($\chi^2 = 0.8, p = 0.6703$).

The results suggest that success of imitations was not dependent on the design condition of the robot: 2 children imitated all signs successfully, while 3 imitated no signs successfully. Half of the children (5), signed between 4-7 signs correctly. Based on this evidence, we are unable to reject the null hypothesis, and we can not accept H2. It is possible that success of imitations is dependent on the child themselves, be it their prior skills, interest level in technology, or other variables related to them. However, this possibility could not be tested with the data available.

While on average children were able to imitate the robot, it is important to note that 3 children achieved no successful imitations. The intervention does not fit all children with autism, and studies in a clinical setting need to be conducted in order to determine how to best conduct this intervention, and who would benefit from it.


\subsection{Attention focus}

Attention focus on a certain subject was calculated as a percentage of the length of each experiment, as times varied significantly between children. Attention focus categories were defined as robot, therapist, companion of the child, or elsewhere. Attention focus was determined by examining the footage of children interacting with the robot, gathered from two cameras. Each second of footage was manually coded for the child's gaze direction, using the software BORIS (Behavioral Observation Research Interactive Software) \cite{boris}. The total interaction time used in the experiment, with the shortest time used for the entire experiment being 367 seconds (just over 6 minutes), and the longest time 1666 seconds (nearly 28 minutes). The variation was due to children's different response speeds. An independent second coder was used to analyze one video to examine agreement levels, of both imitation and gaze direction. The same video was also coded by me a second time to determine internal agreement.

\begin{table}
  \centering
  \renewcommand{\arraystretch}{1.2}
  \begin{tabular}{|p{3cm}|c|c|c|c|c|c|}
    \hline
    \multirow{2}{*}{\textbf{Gaze direction}} & 
    \multicolumn{2}{c | }{\textbf{Sign only}} & 
    \multicolumn{2}{c |}{\textbf{Image}} & 
    \multicolumn{2}{c |}{\textbf{Light}} \\
    %\hline
    % \textbf{Inactive Modes} & \textbf{Description}\\
    \cline{2-7}
    & \textbf{Mean} & \textbf{SD} & \textbf{Mean} & \textbf{SD} & \textbf{Mean} & \textbf{SD} \\
    %\hhline{~--}
    \hline
    Robot & 73.36 \% & 29.89 \% & 74.01 \% & 30.77 \% & 74.77 \% & 29.12 \% \\ \hline
    Therapist & 5.14 \% & 6.07 \% & 5.41 \% & 9.01 \% & 5.81 \% & 11.35 \% \\ \hline
    Companion & 2.95 \% & 3.44 \% & 3.83 \% & 2.97 \% & 3.28 \% & 2.97 \%  \\ \hline
    Elsewhere & 18.55 \% & 25.83 \% & 16.74 \% & 26.47 \% & 16.13 \% & 22.39 \%  \\ \hline
    
  \end{tabular}
 
  \caption{Children's gaze direction relating to three different design conditions.}
  \label{table:attention}
\end{table}

For the data presented in table \ref{table:attention}, a Friedman test revealed no significant effect of design conditions on the focus on the child's attention on the robot, $\chi^2 = 0.97436, p = 0.6144$.

No significant effect of the design condition was noted on the child's attention on the therapist ($\chi^2 = 0.63158, p = 0.7292$), their companion ($\chi^2 = 0, p = 1$), or elsewhere ($\chi^2 = 0.85714, p = 0.6514$).

Based on this evidence, we are unable to reject the null hypothesis, and we can not accept H3.


\begin{table}
  \centering
  \renewcommand{\arraystretch}{1.2}
  \begin{tabular}{|p{3cm}|c|c|}
    \hline
    \multirow{2}{*}{\textbf{Gaze direction}} & 
    \multicolumn{2}{c | }{\textbf{All}}  \\
    %\hline
    % \textbf{Inactive Modes} & \textbf{Description}\\
    \cline{2-3}
    & \textbf{Mean} & \textbf{SD}  \\
    %\hhline{~--}
    \hline
    Robot & 73.89 \% & 29.80 \% \\ \hline
    Therapist & 5.38 \% & 8.41 \% \\ \hline
    Companion & 3.38 \% & 2.79 \%  \\ \hline
    Elsewhere & 17.24 \% & 24.16 \%  \\ \hline
  
  \end{tabular}
  
  \caption{Gaze direction in all design conditions combined.}
  \label{table:attentionAll}
\end{table}

Examining the table \ref{table:attentionAll}, which shows the children's gaze direction among all design conditions, the majority of focus was clearly on the robot, with elsewhere being a second. However, standard deviation is quite high among all gaze attention targets. This means that there is high variation between children. This reinforces the imitation results, in that the effectiveness varies highly by child.


%%%%%%%%%%
%%%%%%%%%%

\section{Qualitative analysis}

Simple surveys containing five questions were conducted with the children by the neuropsychologist after the experiment. 6 children out of the 10 whose data is analyzed in the quantitative section were able to answer these questions. 4 children gave no response to any of the questions. With the children's companions, surveys containing 12 questions were conducted. 8 companions answered these surveys.

As the data obtained from these surveys are not significant in size, no coding for themes was developed beforehand to go through the answers. Instead, all answers were examined for common statements about or evaluations of the design conditions being examined. Additionally, other evaluations or suggestions regarding the robot and its use in this therapeutic context were regarded as valuable data to guide future applications.


\subsection{Survey with children}

All 10 children were asked to answer a short survey after the experiments (survey can be seen in appendix \ref{chapter:children}). 6 children were able to answer the first questions about how the robot felt, and 5 were able to answer about the robot, its lights and images. All children were also given a chance to give any additional comments about the robot, but none of them did.

Out of the options fun, boring and scary, 5 out of 6 children said the robot was fun, although 1 said it was also scary (table \ref{table:kidsFeeling}). 1 child thought the robot was only scary. This indicates a general positive attitude toward the robot from the children, although its scary qualities should be further examined, and removed for future iterations.

\begin{table}
  \centering
  \renewcommand{\arraystretch}{1.2}
  \begin{tabular}{|p{4cm}|c|c|c|c|}
    \hline
     & 
    \textbf{Fun} &
    \textbf{Boring} &
    \textbf{Scary} \\\hline
    %\hline
    \textbf{How did the robot feel to you} & 5 & 0 & 2 \\ \hline
  \end{tabular}
  \caption{All 6 children responded to this question.}
  \label{table:kidsFeeling}
\end{table}

\begin{table}
  \centering
  \renewcommand{\arraystretch}{1.2}
  \begin{tabular}{|p{4cm}|c|c|c|c|}
    \hline
     & 
    \textbf{Good} &
    \textbf{Bad} \\\hline
    %\hline
    \textbf{The robot is} & 5 & 0\\ \hline
    \textbf{The images are} & 5 & 0\\ \hline
    \textbf{The lights are} & 5 & 0\\ \hline
  \end{tabular}
  \caption{5 out of 6 children answered these questions.}
  \label{table:kidsDesign}
\end{table}


All 5 of the kids who were able to answer if the robot, its images and lights were good, said that they all were (table \ref{table:kidsDesign}). These answers further indicate that the children had a generally positive outlook on the robot and its qualities. 


\subsection{Survey with adults}

Out of 10 companions present in the experiments, 8 returned the survey. Some of the participants filled the survey directly after the experiment, and some took it home and turned it in later. The survey contained 12 questions, but not all participants answered all questions. The survey can be seen in appendix \ref{chapter:companions}.

Questions with pre-set answers are presented in tables, while freeform answers are discussed within the text. Quotations are translated from Finnish.


\subsubsection{How did the robot feel?}

\begin{table}
  \centering
  \renewcommand{\arraystretch}{1.2}
  \begin{tabular}{|p{4cm}|c|c|c|c|}
    \hline
     & 
    \textbf{Fun} &
    \textbf{Boring} &
    \textbf{Scary} &
    \textbf{Other} \\\hline
    %\hline
    \textbf{How did the robot feel to the child} & 7 & 1 & 2 & interesting (2), new (1)\\ \hline
    \textbf{How did the robot feel to you} & 7 & 0 & 0 & interesting (5)\\ \hline
  \end{tabular}
  \caption{All 8 companions answered these questions.}
  \label{table:feeling}
\end{table}

The majority of the companions, 7 out of 8, reported that the robot seemed to feel fun to the child, although 2 of them said it also seemed scary (table \ref{table:feeling}). This further indicates that there are some scary qualities of the robot that should be fixed. 1 companion regarded the robot as boring to the child. 

These answers support what the children said they felt about the robot. The perception of the robot was aligned between children and their companions – in both cases where a child answered that the robot felt scary, their companion had the same perception of the child's experience.

The majority of the companions, 7 out of 8, said the robot felt fun also to themselves. 5 of them added that it was interesting in the open answer (table \ref{table:feeling}). This indicates a positive reaction from the companions toward applying the robot in this type of therapeutic use. This is promising, as future applications of the robot, especially in everyday use, would need co-operation from the children's parents and caretakers.


\subsubsection{Design conditions}

\begin{table}
  \centering
  \renewcommand{\arraystretch}{1.2}
  \begin{tabular}{|p{4cm}|c|c|c|c|}
    \hline
     & 
    \textbf{Sign only} &
    \textbf{Image} &
    \textbf{Light}\\\hline
    %\hline
    \textbf{Rank the design conditions} & 5 & 14 & 2\\ \hline
  \end{tabular}
  
  \caption{8 companions answered these questions. The companions were asked to rate each design condition, giving the best one 2 points, the second best 1 point, and the worst 0 points. Some companions only indicated the best condition, in which case this condition was given 2 points.}
  \label{table:designConditions}
\end{table}

Companions of the children were asked to rate the design conditions from best to worst. The best condition was given 2 points, the second rated was given 1 point, and the worst 0 points (table \ref{table:designConditions}). Some companions only indicated what they thought was the best condition. 

1 companion chose ``Sign only" as the best condition. The child of this companion did particularly well in the experiment, and did not need the additional images or lights to stay focused or imitate successfully. This supports the idea that robots should be modular in complexity, and modular per child. 2 companions were concerned whether the voice and sign were sufficient for the child to understand what concept was actually meant by either.

``Image" was overwhelmingly rated the best, by 7 out of 8 participants, with a total of 14 points. Additional positive remarks about the ``Image" design condition were given by 7 out of 8 companions, noting that it ``grabbed the child's attention", and ``helped them understand what was meant by the sign". The image was commented to add to the child's interest, and clarity of the sign.

The ``Light" condition scored only 2 points. 4 out of 8 companions remarked negatively about the light, for example that ``it did not seem to have any effect", or that they ``did not notice it". Most likely, the lights had no significant impact on the interactions with the children in this experiment.


\subsubsection{Contact and usefulness}

\begin{table}
  \centering
  \renewcommand{\arraystretch}{1.2}
  \begin{tabular}{|p{6cm}|c|c|c|c|}
    \hline
     & 
    \textbf{Yes} &
    \textbf{No} \\\hline
    %\hline
    \textbf{Did the child have a connection with the robot} & 8 & 1\\ \hline
  \end{tabular}
  \caption{8 companions answered these questions, 1 answered both yes and no.}
  \label{table:contactYesNo}
\end{table}


\begin{table}
  \centering
  \renewcommand{\arraystretch}{1.2}
  \begin{tabular}{|p{6cm}|p{2cm}|p{2cm}|p{2cm}|}
    \hline
     & 
    \textbf{Better than with human} &
    \textbf{Same as with human} &
    \textbf{Worse than with human}\\\hline
    %\hline
    \textbf{How was the child's connection with the robot} & 2 & 5 & 3\\ \hline
  \end{tabular}
 
  \caption{8 companions answered these questions.}
  \label{table:contact}
\end{table}


All of the companions said the child had a connection with the robot, although 1 also said no (table \ref{table:contactYesNo}). This is a promising result, and supports the result of a mean of 73.89 \% of gaze direction focused on the robot (table \ref{table:attentionAll}), in indicating that the robot was successful in capturing and keeping the children's attention throughout the interaction.

The majority of companions rated the child's contact with the robot as being on a similar level with a human, with 1 also choosing better. 1 chose solely better, and 3 chose solely worse (table \ref{table:contact}). This could possibly indicate that the social interaction partner being specifically a robot may not have any significant advantage over a human, contrary some of the research presented in chapter \ref{chapter:background}, although the sample is so small and the question so general that it is difficult to contest existing research based on this. This could also mean that the robot used in this experiment was too unsophisticated to generate any significant advantages.

\begin{table}
  \centering
  \renewcommand{\arraystretch}{1.2}
  \begin{tabular}{|p{6cm}|c|c|c|c|}
    \hline
     & 
    \textbf{Yes} &
    \textbf{No} \\\hline
    %\hline
    \textbf{Could the child benefit from the robot} & 6 & 3\\ \hline
  \end{tabular}
  \caption{8 companions answered these questions, 1 answered both yes and no.}
  \label{table:benefit}
\end{table}


6 out of 8 companions said the child could benefit from use of the robot, although 1 also said no (table \ref{table:benefit}). 5 out of 8 companions also made additional positive comments about the robots potential in this application, for example that the robot was ``interesting to the child", and that the ``robot's positive feedback encouraged the child". 1 companion regarded the robot as a good potential ``learning platform for contact". This shows that there is a general interest from the children's companions in the robot as a tool for therapeutic use. 

2 out of 8 companions said their child could not benefit from use of the robot. 1 of the children whose companions regarded the robot as not useful did not imitate any signs successfully. The companion remarked that the child was ``slightly suspicious with new people, until trust is achieved". The companion remarked that their child usually established contact through touch, which was not possible in this case. This statement supports the design requirement of modularity per child. The companion who chose yes and no remarked that the robot's movements seemed a bit stiff. They also remarked that their child was used to using the picture symbol system to communicate, rather than signs, which is why the child did not imitate the robot at first.

The other child whose companion regarded it as not useful, performed 7 successful imitations. In this case, the companion of the child was questioning whether the child really connected the sign with the word the robot was saying, since the child does not understand speech well. The parent was unsure whether the child truly understood what concept was being discussed. This raises an important concern for future research: verifying whether the children are understanding the signs, or merely imitating them.


%%%%%%%%%%
%%%%%%%%%%

\section{Results and Implications}

Analysis of the children's imitation attempts indicated no statistically significant effect of the different design conditions. Similarly, analysis of the children's gaze direction indicated no statistically significant effect of the design conditions.

However, the surveys conducted with the children and their companions give an overview of how the children and their companions responded to different design conditions, and the robot in general.

\vspace{3mm}
\noindent\textbf{Children were able to imitate signs performed by the robot}

7 out of 10 children performed a non-zero number of imitations of the robot's signs. This means that the robot has a potential use as an assistive sign language tutor, and there is scope for further study. However, 3 out of 10 children performed zero successful imitations of the robot's signs. This suggests that the robot is not suitable for all autistic children. Further studies should be conducted in order to determine who would benefit from this intervention.


\vspace{3mm}
\noindent\textbf{Positive experience for children, however robot can be scary}

The children's survey answers indicated that they had a mainly positive experience with the robot (tables \ref{table:kidsFeeling} and \ref{table:kidsDesign}), and their companions' answers supported this interpretation (table \ref{table:feeling}). The scary qualities of the robot should be identified and altered for future experiments. 1 parent suggested that the robot's black hands could be seen as scary, or the noise that it makes when it moves. The parent remarked that their child has sensory sensitivity, characteristic of ASDs, which could have led to the servos' noise being scary. The noise from the robot's servos was also regarded as an issue for understandability of the robot's speech by 2 companions, as it sometimes drowned out the speech. For future iterations, the hands' color could be changed, and noise from the robot's servos should be further minimized.

\vspace{3mm}
\noindent\textbf{``Image" was the strongest design condition, according to the perception of the companions}

The children's opinions on the design conditions made no preferences (table \ref{table:kidsDesign}). However, the companion's answers showed a clear preference for the ``Image" design condition, and a non-preference for the ``Light" design condition. In further design of this or other robots, images should be used to support the teaching of signs, and lights should not be used in this realization form.

\newpage
\vspace{3mm}
\noindent\textbf{Forming a connection with the robot was successful}

The companion's answers indicated that the children formed a connection with the robot (table \label{table:contact}), which is supported by the mean of 73.89 \% of children's gaze direction being focused on the robot, with a standard deviation of 29.80 \% (table \ref{table:attentionAll}). This is promising for future solutions, and proves that contact between a robot teaching assistive signs and a child with ASD can be established.

\vspace{3mm}
\noindent\textbf{Robot has potential benefits}

6 companions thought the child could benefit from the robot (table \ref{table:benefit}). 7 companions thought the robot was fun to them, and 5 thought it was interesting (table \ref{table:feeling}). This shows there is a general interest for the robot being used as a learning platform in the future.

\vspace{3mm}
\noindent\textbf{Understanding of signs needs to be verified}

Other concerns that became apparent were that it should be made sure that children connect the words with the sign, and are not merely imitating it. This should be verified in future iterations. According to a parent, ``images help support understanding", which further supports the implementation of the ``Image" design condition in future iterations.

\vspace{3mm}
\noindent\textbf{Performance of signs needs to be improved}

1 parent noted that the signs the InMoov was performing were somewhat stiff. Speech therapist Lehtonen also agreed with this, and said that she noticed the children imitating the robot's stiffness. In order to avoid having children signing too stiffly as a result of the robot, its movements need to be made smoother. Lehtonen also remarked that in future iterations, implementation of facial expressions could be beneficial, in order to better communicate the tone of a statement (A. Lehtonen, personal communication, June 25, 2018).

%\newpage
\vspace{3mm}
\noindent\textbf{Modularity per child}

One of the companions discussed with the psychologist after the experiment that the room had had too much stimulation for the child, which is why they could not focus. The child had been to the same room before, and was expecting certain type of play to happen there, which is common for people with ASD who rely on routines. For future experiments, it should be made sure that the experiment environment is stimulation-free, and that it does not hold any previous connotations for the children who are especially prone to routines.

One parent's preference for no images and no lights, as well as one parent's preference for a different room support that suggestion that the robot should be modular per child, and in complexity. One parent who stopped the experiment due to the fact that their child could not experiment remarked that music could have helped their child focus. If the robot were to be taken into use by specific children, these specific children's preferences should be taken into consideration.
