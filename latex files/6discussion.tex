\chapter{Discussion}
\label{chapter:discussion}

This chapter combines the results discussed in the previous chapter with the theoretical knowledge presented in chapter \ref{chapter:background}, along with how to implement it using the design framework presented in chapter \ref{chapter:design}. This chapter evaluates how effectively the original design defined in chapter \ref{chapter:design} was realized, and makes suggestions for future improvements, both for this particular robot, and other robots built for the use of teaching assistive signs to children with autism. This chapter also answers the research questions, and examines the limitations of this study.

%%%%%%%%%%%
%%%%%%%%%%%

\section{Answers to the research questions}

\vspace{3mm}
\noindent\textbf{RQ1: What are the design choices that may impact the InMoov's usefulness as a sign language tutor?}
\vspace{1mm}

In order to examine the design choices that could impact the InMoov's usefulness as a sign language tutor, a design framework was defined in chapter \ref{chapter:design}. Using the framework, an initial design for the robot, which would be used in the study, was defined. These design choices were guided by five design guidelines defined in chapter \ref{chapter:background}, which were based on previous studies on autism and robots used in therapy with autistic children. The design choices made involved making decisions about the robot's environment, form, interaction and behavior. In order to specifically examine the usefulness of a certain design choice, three different design conditions were designed, which were compared. The design process involved consulting the speech therapist and neuropsychologist as experts.

\vspace{3mm}
\noindent\textbf{RQ2: Is the designed robot successful as an assistive sign tutor?}
\vspace{1mm}

7 out of 10 children performed a non-zero amount of successful imitations of the robot's signs. However, 3 out of 10 children performed zero successful imitations. This suggests that the robot has the potential to be a successful assistive sign tutor to some autistic children. The success of the children's imitation seemed to be dependent upon the specific child, rather than the different design conditions. This leads to the question of whether the success of the outcome is more dependent on the robot's design, or other surrounding factors, such as how the child was feeling at the moment or what their pre-existing skill level was. To determine whether the robot is actually successful as a tutor over time, experiments with multiple measurements are needed.

The robot succeeded in providing a positive experience for the children and their companions (tables \ref{table:kidsFeeling} and \ref{table:feeling}). The robot was generally successful in capturing and retaining the children's attention during the interaction (tables \ref{table:attentionAll} and \ref{table:contactYesNo}). 6 out of 8 children's companions also remarked that their children could benefit from the use of the robot as a tool (table \ref{table:benefit}).

\vspace{3mm}
\noindent\textbf{RQ3: How should the initial design or consequent designs be modified?}
\vspace{1mm}

Out of the original design drivers, the first three (simplicity of form; consistent, structured, simple behavior; and positive, supportive, rewarding experience and environment) were well realized. These should be further improved upon in further designs. 

However, the last two design drivers (modularity of complexity; modularity specific to child's preferences) were not well enough realized, as indicated by children's companions calling for more modularity in the qualitative results. In future realizations of the robot, these design drivers should be emphasized more, with the help of consulting the children's parents or caretakers, who know their individual strengths and preferences. These can then be incorporated into any aspect of the robots design: its environment, form, interaction of behavior.

Out of the design conditions, ``Image" was the best in terms of perceived usefulness, according to the surveys completed with the children's companions. However, quantitative data on imitation success and attention focus did not show any effect of design condition. Due to partial evidence of the design condition's usefulness, it should be further developed in future iterations. The ``Light" condition was not perceived as beneficial by the children's companions, and the lights should be removed in their current form in future iterations.

Specific alterations to the design of this robot that came up in the results were: finding out what makes the robot scary and changing the robot's form accordingly, minimizing the sounds that come from the robot's movement, making the robot's movements smoother, adding music to the robot, and making the robot able to be touched. These suggestions all involve modifying the robots form (as defined in figure \ref{fig:form}). Finding out what exactly makes the robot appear scary needs to be done by comparing different qualities, and receiving feedback from the children and parents. The robot's sounds should be changed: they should have more diversity according to child using it, and the sounds made by its joints moving should be minimized. Its movements should also be modified to appear more organic. Additionally, the robot's tactile sensations could be modified to make it softer and thus more inviting and friendly (although this would need to be compared to the current condition of hard tactile sensations, in order to make an informed design decision). Making the robot be able to interact in a tactile manner could also involve re-designing its interaction dimension (figure \ref{fig:interaction}), so that it could sense and respond to the child's touch. With autistic children, a touch-sensing robot could for example be used to teach the child what kind of physical contact is appropriate.

For the experiments themselves, specific alteration suggestions that came up in results are: make sure the room is free of external stimuli, make sure the room has no previous connotations to the child, have the child spend more time with the robot so that they can trust it, verify that the child is understanding the concept of the word during the experiments, and not only imitating.


%%%%%%%%%%
%%%%%%%%%%

\section{Limitations of the Study}

Due to the pilot nature of the study, there are several limitations. Among these were limitations due to the InMoov, quantity of participants and the experimental set-up. Reliability and validity of the results are also evaluated.


\subsection{Limitations of the InMoov}

The InMoov's software and hardware was not always reliable, and the InMoov did shut down in the middle of one of the experiments, which led to the unusability of the data obtained from it. The InMoov stayed operational during the other experiments. In future experiments, it is worthwhile to consider other robots with more stable builds. 

The InMoov's hardware has also had malfunctions, such as its shoulder dropping out of its socket, but luckily none happened during the experiments. The robot's movements were also noted as stiff by both the speech therapist and a child's companion. The robot does not perform the signs on the same level as a human.

The operation of the InMoov was done by a human observer in real-time, which meant that it did not always react quickly to the child's imitation attempts. In the future, responses could be automated to provide more immediate feedback to the child.

The UI of the InMoov's operating system, MyRobotLab, also provides some limitations. In the future, a symbolic user interface, that could be used by engineer and speech therapist alike, would be beneficial.


\subsection{Quantity of experiment participants}

The quantity of experiment participants was limited, as it is difficult to arrange experiments with children with ASDs. The limited number of participants could have led to the inability to obtain statistically significant effects of design conditions on attention or imitation success. In the future, studies with more participants are needed.

The limited number of participants available also led to the age range of the children being larger than was originally planned. The original planned age range was 6-12, while the realized age range was 11-16, with even one 23 year old present. Unfortunately the two youngest participants, aged 5 and 7, were left out of the data analysis due to incomplete data. In order to develop a robot specifically for children, future studies need to include children from younger age groups as well.

Additionally, at least one of the participants had sensory sensitivity, which others did not. This makes it harder to generalize the results.

The participants were all selected by the neuropsychologist and speech therapist, and thus deemed to be appropriate for the experiment. On this basis, we can take the results of this experiment and apply them to further design of robots in use with children with ASDs in the future.


\subsection{Experimental set-up}

To truly examine whether the children are learning the meaning of the signs, or applying them to their daily language, a longitudinal study would need to be organized. One parent remarked that ``if the child could have been with the robot longer, the signs could have started flowing better". Additionally, this pilot study can only observe how the children respond to the robot in this one instance. To accurately predict what effect long-term therapy with robots could have, a long-term study is needed.

If the robot were to be applied as a therapeutic tool, clinical grade research needs to be done. A study with a control group of neurotypical children, with the robot being compared to a human teacher needs to be conducted.

Additionally, this study only examines one dimension of the robot's design – its interaction schemes. In order to develop a comprehensive solution, other design dimensions should be studied in the future, in relation to the robot's usefulness in this context. Determining which dimensions should be examined should be done together with experts of the domain (such as psychologists or therapists), and the children and their companions. 


\subsection{Reliability}

In this study, inter-rater and test-retest reliability are relevant for the quantitative methods used. To perform reliability checks, one video was retested by another observer, as well as retested by myself.

Cohen's kappa coefficient was calculated to examine the reliability of coding the child's gaze direction from the front and back cameras. 

Between the original coding and a retest, $\kappa = 0,858$, indicating an almost perfect agreement according to Landis and Koch \cite{landis1977measurement}, and excellent agreement according to Fleiss \cite{fleiss2013statistical}. This indicates high test-retest reliability. 

Between the original coding and coding performed by another observer, $\kappa = 0,840$, also indicating an almost perfect agreement according to Landis and Koch, and an excellent agreement according to Fleiss. This indicates high inter-rater reliability.

Between the retest and coding by another observer, $\kappa = 0,905$, which also indicates almost perfect or excellent agreement. This indicates that the method of coding gaze direction from the front and back cameras can be determined as reliable.

Cohen's kappa coefficient was also calculated to examine the reliability of classifying the response of the child to the robot's imitation request. Between the original coding and a retest, $\kappa = 1$, with all classifications of response matching each other perfectly. Between the original coding and coding performed by another observer,  $\kappa = 1$, with all classifications matching each other perfectly. Similarly, for the retest and coding performed by another observer, $\kappa = 1$.

The definition of successful imitation attempts was quite strict. The definition could be broadened in future experiments, to include imitations to be correct also with human help. However, here we wanted to examine purely the robot's effects. 

Inter-method reliability can be examined by comparing the results of the quantitative methods with the qualitative methods. However, as the quantitative methods did not show any significant effect of design condition, inter-method reliability is difficult to examine. The result of children establishing successful contact seemed to be supported by both quantitative and qualitative methods. 8 out of 8 companions said that their child established contact with the robot (table \label{table:contact}, which is supported by the  mean of 73,89 \% of children's gaze direction being focused on the robot.

The methodology of asking children with ASD to give their opinions about a robot is novel to my knowledge, so evaluation of its accuracy and reliability is difficult. It is difficult to know how reliable these answers are: not all of the children understand speech perfectly, so some may have had trouble understanding the questions. Additionally, the answers were recorded by the neuropsychologist and not the children themselves, so they were up to her interpretation. The children did not independently and spontaneously communicate their feelings about the robot. Due to this, the children's replies about the design conditions were not taken as pure statement of opinion, but rather as indicators of a positive attitude toward the robot overall.

The methodology of asking companions of the children to give their opinions via the survey seemed to be more reliable. The replies were consistent throughout the surveys.


\subsection{Validity}

The following is a discussion of the internal and external validity of the experimental methods.


\subsubsection{Internal validity}

History within the experiments themselves is taken into account. The child may learn to sign more accurately with the robot as they progress during the interactions, which may lead to an apparent effect between the design condition and the accuracy. This is mitigated by repeating each design condition three times, in a randomized order for each child.

Instrumentation, or changes in calibration of measurement tools is taken into account. The footage analyzed for data was recorded from three different angles, so that if the child happened to be in a different spot of the room due to the chair moving, they would still not be obscured and their gaze direction could be reliably assessed.

Maturation may have interfered with the experiments, for example by children growing more tired during the experiments. The experiments were all kept as short as possible, with them all being under 30 minutes. However, if children were tired prior to the experiments, this may have affected the results. The effect of conditions prior to therapy on the therapy has been noted in a previous study where autistic children interacted with a robot \cite{robins2004effects}.

Due to the one-time nature of the experiment, no experimental mortality or selection-maturation interaction was observed.


\subsubsection{External validity}

The group selected for the study is not entirely representative due to the low number of participants, and the relatively high age range (12-16, with one 23 year old). 3 out of 10 participants are mildly autistic, and 7 out of 10 are severely autistic. Only 2 out of 10 participants are female. Due to ASD having varying presentation in different people, participants could not in any case be representative of the entire ASD child population. Further testing is required with a more widely varied participant group gender and autism severity wise, and with younger children.

Selection biases for successful outcome among the participants are assumed to not be present, as the participants were selected by two professionals, the neuropsychologist and the speech therapist.

As only one experiment was conducted, there is no risk of a pretest affecting the outcome, and no multiple-treatment interference.

%%%%%%%%%%
%%%%%%%%%%

\section{Practical implications}

The practical objective of this thesis was to examine the success of a specific robot in teaching assistive signs to children with autism. The goal was to create an initial design in a systematic way, and use it for pilot tests. The practical contribution of this thesis is the framework developed for designing social robots for specific uses, and the design recommendations made upon the examination of the completed design. The design recommendations contribute to the design of future robots for the use of teaching assistive sign language with children with autism, and the design framework contributes to this, and also to the method of designing social robots in general. Additionally, the developed research methods can be used in future research.

This thesis focused on a single design iteration, in which professionals of the field (a neurospychologist and a speech therapist, as well the robotics engineer writing this thesis) were involved in defining the problem space and design drivers, and children and their companions were involved in evaluating the interaction of the solution space. I propose that for robotic solutions to be fully effective, children with ASD and their companions should be involved in all aspects of the robot design: defining the problem space and design drivers, as well as defining the solution space: the robot's environment, form, interaction and behavior. The involvement of these co-creators should ideally be done through each design phase, and in an iterative manner when building the robot.

Should the research continue with this InMoov as a solution platform, the next iteration would be modified based on the feedback obtained through the qualitative surveys (as the quantitative data provided no actionable outcomes). Design iterations should also be done on a child-to-child basis, as modularity per child was underlined in the qualitative results.

The method of using gaze direction as a measure of attention focus can be considered as particularly reliable. The classification of the children's reactions were also reliable. These methods could be used in future research when data being examined is video material of a child interacting with a robot. The qualitative methods, having children and their companions fill in surveys, need to be further developed in future research, but show promise as a method of examining both the success of robot, and as a method of involving users and their loved ones in the design process. Finding a method of involving the children especially is a worthwhile goal for future research. The design framework's iterative process should be used to gather user feedback from children. The novel survey presented here can be used as a basis for developing user feedback methods to be used specifically with children with ASDs.

%%%%%%%%%%
%%%%%%%%%%

\section{Theoretical implications}

The theoretical objective of this thesis was to complete the first pilot study of a robot used to teach assistive signs to children with autism. Previous research has examined the use of robots in communication therapy with children with ASDs, and to teach sign language to neurotypical children. This previous research was examined to create a basis for the initial design of the robot. This thesis is the first study in a new area of robotics research, which researchers can take into consideration in future studies. 

A study that examined the use of a robot as a teacher of sign language to children served as the original inspiration for this application of the robot. The study called for continued research into creating social robots for use by disabled or impaired people. The researchers assert that people such as hearing-impaired children and children with ASD need the assistance of technology to survive more than the majority of society \cite{uluer2015new}.

A comparative design methodology was selected, as a need for research into the design spaces of behavior, appearance and interactional competencies of robots in autism therapy had been previously identified \cite{designSpaces, robins2006appearance}. Comparative design studies were especially cited as an efficient tool to help identify and meet the needs of this group of users \cite{robins2006appearance}. This informed the development of the design framework and research performed here.

The design framework introduced in chapter \ref{chapter:design} is the biggest contribution of this thesis. To my knowledge, it is the only design framework that aids the entire design process of a social robot. 

In order to showcase how the design framework relates to other existing robots, the robots Charlie and Keepon, which were introduced in chapter \ref{chapter:background}, are situated into the framework. This is detailed in the appendices \ref{chapter:charlieDesign} and \ref{chapter:keeponDesign}. Situating these robots into the framework acts as a demonstration of its usability outside the context of the particular implementation introduced in this thesis.


%%%%%%%%%%
%%%%%%%%%%

\section{Future research}

The study in this thesis tested only one dimension of the design space (interaction schemes), but can be treated as an example of how to use the framework for experiment design in general. Future research should focus on varying other design dimensions in order to further improve the design of robots to be used for assistive sign language tutoring. Future research should also focus on determining who exactly would benefit from this type of intervention, as the success level of imitations varied by child.

Future research should explore the possibility of the robot as a tool in long-term use. This requires longitudinal clinical studies, with a control group of neurotypical children and with a comparison to a human as a teacher. This would also provide the possibility of examining whether children are learning the signs long term, and if they are transferring learned skills into everyday life. Clinical research in the field of using robots in communication therapy with autism is lacking \cite{Begum2016}. Robot-mediated interventions need to be proved to produce stable positive effects, to be considered evidence-based practices, and therefore effective clinical interventions \cite{Begum2016}.

If the clinical studies lead to robots being applied as therapy tools, future research should also target the development of an interface for the robot that therapists and psychologists could use to construct and program therapy sessions. The interface should be implemented to any future iterations of the robot introduced in this thesis, as well as other robots intended for a similar use. The interface should be understandable and adaptable to sudden changes. An interface to modify the robot's behavior and interactions would be beneficial to support the modularity of its complexity and modularity per child. 

Future studies should also examine the possibility of placing an assistive sign teaching robot into a child's daily life, for example at their home or at their school. This would make the robot more accessible to the child, and provide more opportunities for the child to practice their skills with it. This would also remove some of the workload of continued practice from the child's loved ones. It would also have the advantage of being freely accessible by the child,  which would increase the spontaneity of interactions, and aid generalization. The robot could use its internal memory to remember signs that are unique to a child, providing continuity for the child to use their individualistic signs. 

Possible future research regarding the design framework could be creating more precise methods for describing the robot's voice, sounds, olfactory and tactile sensations, which were examined in the ``form" section (\ref{chapter:form}). In future versions of the framework, new design dimensions should also be implemented. Three dimensions that will be important in the future design of social robots are perception, personality, and mobility. These dimensions came out in discussions with experts after the study was conducted, which is why they are not included in the framework. This fact also indicates that as new knowledge is accumulated about social robotics, the framework should be modified accordingly.

Perception refers to the ability of the robot to perceive its surroundings, and make decisions based on these. In the future, robots could perceive not only simple speech, but use voice prosody analysis and facial analysis to recognize for example the emotions of their interaction partner. Computer vision could aid robots in recognizing objects, and interacting with them. There are other perception technologies that could be implemented in robots of the future.

Personality refers to the robot's manner of interacting. Currently, social robots do not display much character. Robot's interactions currently focus more on task-oriented goals, rather than exploratory ones (this was explored in \ref{chapter:interaction}). As social robots become more sophisticated, users may perceive them more as interaction partners who they can relate to and who can provide meaningful conversation, rather than only complete tasks. In this case, personality of the robot begins to become more important.

Mobility affects the entire design of the robot. Mobility of social robots provides them a larger environment, and modifications to its form in order to accomodate mobility. Mobility can provide entirely new modes of interaction. It also means adjusting the robot's behavior to move in the world, which includes things such as social mobile behavior, such as collision avoidance. Introducing mobility also requires sophisticated perception, in order to avoid safety issues.

Other dimensions may emerge as social robots continue to be developed. The framework is not intended to be static, and should be continuously developed to remain a relevant tool in the future design of social robots.
