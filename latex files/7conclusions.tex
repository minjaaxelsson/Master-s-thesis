\chapter{Conclusion}
\label{chapter:conclusion}


This thesis introduced a social robot that was used to teach assistive sign language to children with autism. The phenomenon of autism in children and the problems associated with it were discussed, as well as assistive sign language as a communication tool for autistic children. Robots that had been previously used in communication therapy use with autistic children, and robots that had been used to teach neurotypical children sign language, were examined. Based on the findings of this research, five design guidelines for the design of social robots to be used with autistic children were defined. 

In order to systematically design a social robot to be used to teach assistive sign language to children with autism, a design framework for the design of social robots was introduced. The design framework is a tool that can be used by experts of different fields, collaborating to create a social robot. Collaboration across domains is often needed in the robotics field, as was discussed in the introduction. Social robotics pioneer Cynthia Breazeal has specifically called for a multidisciplinary approach to robotics \cite{Breazeal2008}. In this case two experts, a speech therapist and a neuropsychologist, contributed to the design decisions with their expertise. The framework acted as a facilitator of effective communication, and as a boundary object for materializing ideas. It introduced a common language for experts from different domains, which is essential for designing social robots that serve their purpose efficiently. The framework also introduced safety and ethical considerations into the design process of robots. This is important, as robots have the capacity to affect human emotions, and can thus pose a risk for the user.

For the case described in this thesis, the open source robot InMoov was modified in order to follow the established design guidelines, and the advice of the experts. The modified robot was tested with 10 autistic children, with its interaction design dimension being examined in a comparative design study. The children's interactions with the robot were filmed, which were analyzed for imitation success and attention direction. Children and their companions also answered surveys on their opinions about the robot. 

Quantitative results obtained from the footage indicated that children were in general successful in imitating the robot. Children were also generally successful in establishing contact with the robot. Measures extracted from the footage were compared across three different design conditions, in order to examine whether the design condition had an effect on the success of imitation or gaze direction. There was no significant effect of design condition on either success of imitation or gaze direction. Qualitative results indicated that the children enjoyed interacting with the robot, and that the majority of their companions regarded the robot as a potentially useful platform for their children to practice assistive signs with in the future. One design condition was a clear favorite among companions of the children, which should influence the design of future robots for similar uses.

The first important contribution of this thesis is a moral one. In the case described here, highly sophisticated technology is used where it creates the greatest benefit: with people whose lives are rendered more difficult by the circumstances of their birth. This thesis serves as proof that rapid prototyping methods can be applied to effectively bring this type of technology to marginalized people, without years of development. 

The most important contribution of this thesis is the design framework. It is the first social robot design framework to describe the entire design process, and it can be an important tool for unifying the design process of future social robots. Previously, design choices may have been made without deliberation, without the designer even consciously realizing that a choice is being made. This framework aids in making design decisions explicit and deliberate. Design considerations in social robotics become increasingly important as robots are being edged closer to humans, embedded in the fabric of our everyday life. The framework facilitates ethical design that takes into account the users' needs. Such design is essential for successfully integrating social robots into our society while maximizing positive impact and minimizing negative effects. In the future, the design framework should be updated and adapted according to the evolving field of social robotics.




